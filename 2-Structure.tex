\subsection{Organizational Structure}
\label{sec:hierarchy}

\subsubsection{\code{PartType} Dataframes}
\label{sec:df}
Data for each particle type (e.g., dark matter, gas, etc.) is stored in a separate \code{PartType} dataframe and indexed by particle ID number. 
Individual fields can be loaded into the dataframes and deleted at will, with coherent slicing across multiple data fields, courtesy of \code{pandas}.  
The base \code{PartType} objects, \code{PartTypeNbody} and \code{PartTypeSPH}, are standard \code{pandas} dataframes with additional functionality for loading standard \textsc{gadget} particle fields from HDF5, converting units, and performing coordinate transformations.  
As such, \code{PartType} dataframes retain all the capabilities of the \code{pandas.DataFrame} from which they inherit, and any operation that creates a new dataframe instance returns a standard \code{pandas} dataframe.

Nonstandard particle fields can be easily loaded into \code{gadfly} as well; fields loaded in this manner simply lack automated unit conversion.
Alternatively, a custom dataframe class inheriting from \code{PartTypeNbody} or \code{PartTypeSPH} as appropriate can be defined to provide methods for loading both nonstandard particle fields and additional derived properties (e.g., temperature). An example of such a custom class---\code{PartTypeCustom.py}---is provided in the examples distributed with \code{gadfly}, and the usage of such a custom class is demonstrated in Figure \ref{fig:usage_example}.

\subsubsection{\code{Snapshot}}
\label{sec:snap}
Each \code{Snapshot} object represents a single HDF5 snapshot file on disk.  File access---provided by \code{h5py}---is handled via the \code{Snapshot} object, and the actual particle data, loaded as needed into the \code{PartType} dataframes described in Section \ref{sec:df}, is gathered here with each particle type contained in a separate \code{PartType} dataframe.  
The information contained in the \textsc{gadget} header is also kept here in a \code{Header} object, along with access to the additional metadata and unit information stored in the relevant \code{Simulation} object.

\subsubsection{\code{Simulation}}
\label{sec:sim}
Metadata relevant to the simulation as a whole, such as filepaths and unit information, are stored in a \code{Simulation} object.  Initializing a \code{Simulation} object is the first step in any analysis using \code{gadfly} as this is where default values for all subsequent analysis are set, including locating all relevant snapshot files, choosing a coordinate system, and setting the field names of the default particle properties expected by \code{gadfly}.  \code{Snapshot}s are loaded via \verb|Simulation.load_snapshot()|, and the parallel batch processing functionality described in Section \ref{sec:parallel} is implemented at this level as well.