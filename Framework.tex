\section{A Framework built on pandas}
\label{framework}

The motivation for building an analysis framework on top of pandas is twofold.  Most importantly, doing so allows us to leverage existing capabilities of that library, and its strong developer community.  Second, using pandas.DataFrame as the primary data container rather than numpy arrays makes it much easier to keep different particle properties indexed correctly while still affording the flexibility to load and remove particle properties from memory at will.

\subsection{Data Organization}
\label{hierarchy}
The framework makes some assumptions about how the simulation output is organized.  First and foremost, that it is in an HDF5 file.  Then, each particle type is expected to be contained in a different group, labeled PartType0, PartType1, etc.  Finally, an additional group, Header, is also expected, containing metadata for the simulation snapshot as hdf5 attributes.  

Data for each particle type (e.g., dark matter, gas, etc.) is stored in a separate DataFrame and indexed by particle number.  Individual fields can be loaded into the dataframes and deleted at will.

\subsection{Units and physical constants}
\label{units}

\subsection{Coordinate Transformations}
\label{coordinates}