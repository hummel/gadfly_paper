\section{Introduction}
\label{sec:intro}

In the past decade, astrophysical simulations have increased dramatically in both scale and sophistication, and the typical size of the datasets produced has grown accordingly.  
However, the software tools for analyzing such datasets have not kept pace, such that one of the primary barriers to exploratory investigation is simply manipulating the data.  
This problem is particularly acute for users of the popular smoothed particle hydrodynamics (SPH) code \textsc{gadget} \citep{SpringelYoshidaWhite2001,Springel2005}.  
Both \textsc{gadget} and \textsc{gizmo} \citep{Hopkins2015}, which uses the same data storage format, are widely used to investigate a range of astrophysical problems; unfortunately this also leads to fractionation of the data storage format as each research group modifies the output to suit its needs.
This state of affairs has historically forced significant duplication of effort, with individual research groups separately developing their own unique analysis scripts to perform similar operations.

Fortunately, the issue of data management and analysis is not endemic to astronomy, and the resulting overlap with the needs of the broader scientific community and the industrial community at large provides a large pool of scientific software developers to tackle these common problems.
In recent years, this broader community has settled on python as its programming language of choice due to its efficacy as a `glue' language and the rapid speed of development it allows.  
This has led to the development of a robust scientific software ecosystem with packages for numerical data analysis like \code{numpy} \citep{VanderWaltColbertVaroquaux2011}, \code{scipy} \citep{JonesOliphantPeterson2001}, \code{pandas} \citep{McKinney2010}, and \code{scikit-image}; \code{matplotlib} \citep{Hunter2007} and \code{seaborn} for plotting; \code{scikit-learn} for machine learning, and statistics and modeling packages like \code{scikits-statsmodels}, \code{pymc}, and \code{emcee} \citep{Foreman-Mackeyetal2013}.

Python is quickly becoming the language of choice for astronomers as well, with the Astropy project \citep{Robitailleetal2013} and its affiliated packages providing a coordinated set of tools implementing the core astronomy-specific functionality needed by researchers. 
Additionally, the development of flexible python packages like \code{yt} \citep{Turketal2011a} and \code{pynbody} \citep{Pontzenetal2013}, capable of analyzing and visualizing astrophysical simulation data from several different simulation codes, have greatly improved the ability of computational researchers to perform useful, insight-generating analysis of their datasets.

Recently, the scientific python community has begun to converge on the \code{DataFrame} provided by the high-performance \code{pandas} data analysis library as a common data structure. 
As a result, once data is loaded into a \code{DataFrame}, it becomes much easier to take advantage of the powerful analysis tools provided by the broader scientific computing ecosystem.
With this in mind, we present a \code{pandas}-based framework for analyzing \textsc{gadget} simulation data, \textsc{gadfly}: the \textsc{GAdget} DataFrame LibrarY.
Rather than providing an alternative to the existing \code{yt} and \code{pynbody} projects, the aim of the \code{gadfly} project is to ease interoperability with the python ecosystem at large, lowering the barrier for access to the tools created by this broader community.

In this paper we present the first public release (v0.1) of \code{gadfly}, which is available at \code{http://github.com/hummel/gadfly}.
The framework design and organizational structure are outlined in Section \ref{sec:framework}, followed by a description of the included SPH particle rendering  in Section \ref{sec:vis}.  Our plans for future development are outlined in Section \ref{sec:future}, and a summary is provided in Section \ref{sec:summary}.