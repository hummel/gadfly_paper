\section{Summary}
\label{sec:summary}
In this paper we have presented the first public release of \code{gadfly}.  We have described the framework's structure, which is designed around three core abstractions: the \code{PartType} dataframe (Section \ref{sec:df}), the \code{Snapshot} object (Section \ref{sec:snap}), and the \code{Simulation} object (Section \ref{sec:sim}).  Additional functionality includes intelligent memory management and file access (Section \ref{sec:fileIO}), basic unit handling (Section \ref{sec:units}), coordinate transformations (Section \ref{sec:coords}), utilities for parallel batch processing (Section \ref{sec:parallel}), and SPH particle visualization (Section \ref{sec:vis}).

\code{Gadfly} is fully open-source, is released under the MIT License, and can be downloaded and installed from its repository at \code{http://github.com/hummel/gadfly}.  Users can submit bug reports via GitHub, and if they know how to fix them, are welcome to submit pull requests.


%\clearpage