\section{A Framework built on pandas}
\label{sec:framework}

There are several motivations for building an analysis framework around the \code{pandas DataFrame}. 
The guiding principle underlying the design of this framework is to enable exploratory investigation.
This requires both intelligent memory management for handling out-of-core datasets, and a robust indexing system to ensure that particle properties do not become misaligned in memory.
Using  the \code{pandas DataFrame} as the primary data container rather than separate \code{numpy} arrays makes it much easier to keep different particle properties indexed correctly while still affording the flexibility to load and remove data from memory at will.
In addition, \code{pandas} itself is a thoroughly documented, open-source, BSD-licensed library providing high-performance, easy-to-use data structures and analysis tools, and has a strong community of developers working to improve it.  
More broadly, as \code{pandas} is becoming the de-facto standard for data analysis in python, doing so simplifies interoperability with the rest of the tools in the ecosystem.

\code{Gadfly} is designed for use with simulation data stored in the HDF5 format \citep{hdf5}.  
While we otherwise aim to keep \code{gadfly} as general as possible, some assumptions about the storage format are necessary.
Each particle type is expected to be contained in a different HDF5 group, labeled \code{PartType0, PartType1}, etc; a \code{Header} group is also expected, containing metadata for the simulation snapshot as HDF5 attributes. 
All particles are expected to have the following fields defined: particle IDs, masses, coordinates, and velocities.  
SPH particles are additionally expected to have a smoothing length, density, and internal energy.  

This section provides an overview of the design and capabilities of the \code{gadfly} framework.  
Section \ref{sec:hierarchy} describes the \code{Simulation}, \code{Snapshot}, and \code{PartType} dataframe objects at the core of \code{gadfly}.  
Our memory-efficient approach to data access is outlined in Section \ref{sec:fileIO}, and our handling of unit conversions and coordinate transformations is discussed in Sections \ref{sec:units} and \ref{sec:coords}, respectively.
