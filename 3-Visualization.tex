\section{Visualization}
\label{sec:vis}
Visualization plays a major role in any analysis of simulation data, and the ability to directly visualize the spatial structure and evolution of a system is often crucial to generating insight.
While the guiding principle of the \code{gadfly} project is to avoid reinventing the wheel, instead relying on the tools developed by the broader scientific python community wherever possible, SPH particle rendering is one critical area where that broader ecosystem proves insufficient.  
To mitigate this shortcoming, \code{gadfly} includes a minimal library of visualization tools.

The particles in an SPH simulation are best thought of as fluid elements sampling the continuum properties of the gas they represent \citep{Lucy1977,GingoldMonaghan1977,Monaghan1992,Springel2010}.  They accomplish this by serving as Lagrangian tracers over which the continuum properties are interpolated using a smoothing kernel $W$. While it is possible to use alternative kernels, most modern SPH implementations (including \textsc{gadget}) utilize a cubic spline kernel \citep{Springel2014}: 
\begin{equation}
W(r,h) =
     \begin{cases}
       1 - 6 \left( \frac{r}{h} \right)^2, &\quad\text{if RMSD}_\text{s-open}\ge6, \text{RMSD}_\text{closed}\ge6\\
       \text{closed,} &\quad\text{if RMSD}_\text{closed}\le2 \\
       \text{semiopen,} &\quad\text{if RMSD}_\text{s-open}\le2\\
     \end{cases}

\end{equation}

The SPH particle rendering algorithm at the core of \code{gadfly}'s visualization tools is designed for projecting three-dimensional gas density distributions down to a two-dimensional image; an example of such a visualization produced by gadfly is shown in Figure \ref{fig:vis}.

\code{Gadfly} includes three separate implementations of this algorithm, each of which is best suited to different conditions:
\begin{enumerate}
\item The primary implementation is written in \code{C}, parallelized using OpenMP, and wrapped in python using \code{scipy.weave}.  This method must be locally compiled, and will fail if the python interpreter cannot locate a \code{C} compiler, or if the requisite libraries are not installed.  However, it its the most powerful implementation, best suited to rendering many particles, and to machines with many processors available.
\item A second implementation makes use of \code{numba} to perform just-in-time (JIT) compilation of pure python code using the LLVM compiler infrastructure.  The resulting serial routine is highly optimized, providing performance within a factor of two of the parallel method on a 16-core machine.  As such, this method is preferable on smaller machines with fewer processors, and for visualizations including fewer particles where the additional overhead associated with the parallel implementation is significant.
\item The final implementation is a pure python routine included only for situations where the other methods have unmet dependencies. This implementation is over 500 times slower than the others, and as such is suitable only to visualizing small numbers of particles, or as a last resort.
\end{enumerate}

These methods are all available as independent library functions, and can be used both with particle data in a \code{PartType} dataframe or independently from the rest of the \code{gadfly} framework, with data stored in \code{numpy} arrays.  
Future versions of \code{gadfly} will greatly simplify the visualization of data stored in a \code{PartType} dataframe through the use of a \code{.visualize()} method (see Section \ref{sec:future}).